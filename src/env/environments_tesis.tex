% Crea una sección de referencias solo para bibtex
\newenvironment{references}{
	\ifthenelse{\equal{\stylecitereferences}{bibtex}}{ % Verifica configuraciones
	}{
		\throwerror{\references}{Solo se puede usar entorno references con estilo citas \noexpand\stylecitereferences=bibtex}
	}
	\phantomsection
	\addcontentsline{toc}{chapter}{\namereferences}
	\begin{thebibliography}{} % Inicia la bibliografía
		\ifthenelse{\equal{\bibtextextalign}{justify}}{ % Formato ajuste de línea
		}{
		\ifthenelse{\equal{\bibtextextalign}{left}}{
			\raggedright
		}{
		\ifthenelse{\equal{\bibtextextalign}{right}}{
			\raggedleft
		}{
		\ifthenelse{\equal{\bibtextextalign}{center}}{
			\centering
		}{
			\throwbadconfig{Ajuste de linea referencias bibtex desconocido}{\bibtextextalign}{justified,left,right,center}}}}
		}
	}
	{
	\end{thebibliography}
}

% Crea una sección de resumen
%	#1	Tabla resumen
%	#2	Título de la tesis
%	#3	Título de la sección
%	#4	Etiqueta del marcador del pdf
\newenvironment{abstractenv}[4]{
	\clearpage%
	\ifthenelse{\equal{\addemptypagespredoc}{true}}{%
		\coretriggeronpage{\insertemptypage}{}%
	}{}%
	\emptyvarerr{\abstractenv}{#1}{Tabla resumen no definida}
	\emptyvarerr{\abstractenv}{#2}{Titulo tesis no definido}
	\emptyvarerr{\abstractenv}{#3}{Titulo seccion no definida}
	\emptyvarerr{\abstractenv}{#4}{Etiqueta marcador del pdf}
	% Añade a los marcadores
	\ifthenelse{\equal{\addabstracttobookmarks}{true}}{
		\phantomsection
		\pdfbookmark{#3}{#4}}{
	}
	% Inserta la tabla resumen
	\ifthenelse{\equal{\showabstracttable}{true}}{
		\begin{flushright}
			\small
			#1
		\end{flushright}
		\vspace{0.75\baselineskip}
	}{
		\vspace*{0.5\baselineskip}
	}
	% Título
	\begin{center}
		\textcolor{\sectioncolor}{\MakeUppercase{\textbf{#2}}}
	\end{center} \newp
	\ifthenelse{\equal{\showabstracttable}{true}}{
		\vspace{-\baselineskip}
	}{
		\vspace{-0.5\baselineskip}
	}
	}{%
	\vfill\null%
}
% Llama al entorno de resumen
\newenvironment{abstractd}{
	\begin{abstractenv}{\abstracttable}{\documenttitle}{\nameabstract}{abstractbookmark}
	}{
	\end{abstractenv}
}

% Crea una sección de dedicatoria
\newenvironment{dedicatory}{
	\clearpage%
	\ifthenelse{\equal{\addemptypagespredoc}{true}}{%
		\coretriggeronpage{\insertemptypage}{}%
	}{}%
	\null
	\phantomsection
	% \ifthenelse{\equal{\adddedictobookmarks}{true}}{
	% 	\pdfbookmark{\nameadedic}{contents}}{
	% }
	\vspace{\stretch{1}}
	\begin{flushright}
		\itshape}{
	\end{flushright}
	\vspace{\stretch{2}}
	\null
}

% Crea una sección de agradecimientos
\newenvironment{acknowledgments}{
	\clearpage%
	\ifthenelse{\equal{\addemptypagespredoc}{true}}{%
		\coretriggeronpage{\insertemptypage}{}%
	}{}%
	\chapter*{\nameagradec}
	\ifthenelse{\equal{\addagradectobookmarks}{true}}{
		\phantomsection
		\pdfbookmark{\nameagradec}{acknowledgments}}{
	}
	\forceindent
	}{
}

% END