% Template:     Auxiliares LaTeX
% Documento:    Funciones exclusivas de Template-Auxiliar
% Versión:      DEV
% Codificación: UTF-8
%
% Autor: Pablo Pizarro R.
%        pablo@ppizarror.com
%
% Manual template: [https://latex.ppizarror.com/auxiliares]
% Licencia MIT:    [https://opensource.org/licenses/MIT]

% COMPILACION
% - environments.tex -> {references, anexo, columna}

% Configuraciones base
\def\anumsecaddtocounter {false}
\newcounter{totalquestions}
\setcounter{totalquestions}{1}

% Insertar nuevo título de pregunta
%	#1	Título
\newcommand{\newquestion}[1]{%
	\emptyvarerr{\newquestion}{#1}{Titulo pregunta no definido}%
	\sectionanum{#1}%
}

% Insertar nuevo título de pregunta encerrado en un recuadro
%	#1	Número pregunta
\newcommand{\newboxquestion}[1]{%
	\emptyvarerr{\newquestion}{#1}{Titulo pregunta no definido}%
	\phantomsection%
	\newp \fbox{\ \textbf{#1}.-\ } \noindent%
	\pdfbookmark{#1}{question.\thetotalquestions}%
	\stepcounter{totalquestions}%
}

% Crea un entorno para definir el tamaño de bloque
%	#1	Tamaño de fuente en pt
\newenvironment{fontsizeblock}[1][\documentfontsize]{
	\changefontsizes{#1 pt}
}{
	\changefontsizes{\documentfontsize pt}
}

% Crea una sección de imágenes múltiples
%	#1	Label (opcional)
%	#2	Caption
\newenvironment{images}[2][]{%
	% Modifica globales
	\def\envimageslabelvar {#1}%
	\def\envimagescaptionvar {#2}%
	\def\GLOBALenvimageinitialized {true}%
	\def\GLOBALenvimageadded {false}%
	
	% Configura caption y márgenes
	\corevspacevarcm{\marginimagetop}%
	\setcaptionmargincm{\captionmarginmultimg} % Eso es para los wrapfig
	
	% Inicia la figura
	\begin{samepage}%
	\begin{figure}[H] \centering%
		\corevspacevarcm{\marginimagemulttop}%
		}{%
		\setcaptionmargincm{\captionlrmargin}%
		\ifthenelse{\equal{\envimagescaptionvar}{}}{ % \ifx\hfuzz no sirve
			\corevspacevarcm{\captionlessmarginimage}%
		}{%
			\corevspacevarcm{\captionmarginimages}%
			\caption{\envimagescaptionvar\envimageslabelvar}%
		}%
	\end{figure}

	% Restablece caption y márgenes
	\setcaptionmargincm{\captionlrmargin}%
	\corevspacevarcm{\marginimagebottom}%
	\end{samepage}
	
	% Restablece globales
	\def\GLOBALenvimageinitialized {false}%
}

% Crea una sección de imágenes múltiples completa dentro de un multicol
%	#1	Label (opcional)
%	#2	Posición de la imagen, "bottom", "top"
%	#3	Caption
\newenvironment{imagesmc}[3][]{%
	% Modifica globales
	\def\envimageslabelvar {#1}%
	\def\envimagesmcpos {#2}%
	\def\envimagescaptionvar {#3}%
	\global\def\GLOBALenvimageinitialized {true}%
	\global\def\GLOBALenvimageadded {false}%
	\checkinsidemulticol%
	\checkoutsideappendix%
	
	% Configura caption y márgenes
	\setcaptionmargincm{\captionmarginmultimg} % Eso es para los wrapfig
	
	% Inicia la figura
	\ifthenelse{\equal{#2}{bottom}}{%
		\begin{figure*}[b] \centering%
	}{%
	\ifthenelse{\equal{#2}{top}}{%
		\begin{figure*}[t] \centering%
	}{%
		\errmessage{LaTeX Warning: Posicion de imagen invalida, valores esperados: bottom,top}
		\stop
	}}%
		\corevspacevarcm{\marginimagemulttop}%
	}{%
		\setcaptionmargincm{\captionlrmargin}%
		\ifthenelse{\equal{\envimagescaptionvar}{}}{%
			\corevspacevarcm{\captionlessmarginimage}%
		}{%
			\ifthenelse{\equal{\captionmarginimagesmc}{0}}{}{\corevspacevarcm{\captionmarginimagesmc}}%
			\caption{\envimagescaptionvar\envimageslabelvar}%
		}%
	\end{figure*}
	
	% Restablece caption y márgenes
	\setcaptionmargincm{\captionlrmarginmc}%
	
	% Restablece globales
	\global\def\GLOBALenvimageinitialized {false}%
}

% END