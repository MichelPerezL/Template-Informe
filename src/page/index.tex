% Template:     Informe LaTeX
% Documento:    Índice
% Versión:      DEV
% Codificación: UTF-8
%
% Autor: Pablo Pizarro R.
%        pablo@ppizarror.com
%
% Manual template: [https://latex.ppizarror.com/informe]
% Licencia MIT:    [https://opensource.org/licenses/MIT]

\newcommand{\templateIndex}{
	
	% -------------------------------------------------------------------------
	% Crea nueva página y establece estilo de títulos
	% -------------------------------------------------------------------------
	\clearpage%
	\begingroup%
	\sectionfont{\color{\indextitlecolor} \indexsectionfontsize \indexsectionstyle \selectfont}
	
	% -------------------------------------------------------------------------
	% Salta de página si está imprimiendo por ambas caras
	% -------------------------------------------------------------------------
	\ifthenelse{\equal{\GLOBALtwoside}{true}}{%
		\coretriggeronpage{\emptypagespredocformat}{}%
	}{}
	
	% -------------------------------------------------------------------------
	% Añade la entrada del índice a los marcadores del pdf
	% -------------------------------------------------------------------------
	\ifthenelse{\equal{\addindextobookmarks}{true}}{
		\phantomsection
		\belowpdfbookmark{\nameltcont}{contents}}{
	}
	\tocloftpagestyle{fancy}
	
	% -------------------------------------------------------------------------
	% Configuración del punto en índice
	% -------------------------------------------------------------------------
	% Agrega los puntos
	\def\cftchapaftersnum {\charaftersectionnum}
	\def\cftsecaftersnum {\charaftersectionnum}
	\def\cftsubsecaftersnum {\charaftersectionnum}
	\def\cftsubsubsecaftersnum {\charaftersectionnum}
	\def\cftsubsubsubsecaftersnum {\charaftersectionnum}
	% Modifica los márgenes
	\ifthenelse{\equal{\charaftersectionnum}{}}{}{
		\def\cftsecnumwidth {1.9em}
		\def\cftsubsecnumwidth {2.57em} % Incremento 0.67
		\renewcommand\cftsubsubsecnumwidth{3.35em} % Incremento 0.78
		\setlength{\cftsubsecindent}{1.91em}
		\setlength{\cftsubsubsecindent}{4.48em} % Incremento 2.57
	}
	
	% -------------------------------------------------------------------------
	% Configuración carácter número de página
	% -------------------------------------------------------------------------
	\renewcommand{\cftdot}{\charnumpageindex}
	
	% -------------------------------------------------------------------------
	% Configuración del punto en número de objetos
	% -------------------------------------------------------------------------
	\def\cftfigaftersnum {\charafterobjectindex\enspace} % Figuras
	\def\cftsubfigaftersnum {\charafterobjectindex\enspace} % Subfiguras
	\def\cfttabaftersnum {\charafterobjectindex\enspace} % Tablas
	\def\cftlstlistingaftersnum {\charafterobjectindex\enspace} % Códigos fuente
	\def\cftmyindexequationsaftersnum {\charafterobjectindex\enspace} % Ecuaciones
	
	% -------------------------------------------------------------------------
	% Desactiva los números de línea
	% -------------------------------------------------------------------------
	\ifthenelse{\equal{\showlinenumbers}{true}}{
		\nolinenumbers}{
	}
	
	% -------------------------------------------------------------------------
	% Cambia tabulación índice de objetos para alinear con contenidos
	% -------------------------------------------------------------------------
	\ifthenelse{\equal{\objectindexindent}{true}}{
		\setlength{\cfttabindent}{1.9em} % Tablas
		\setlength{\cftfigindent}{1.9em} % Figuras
		\setlength{\cftsubfigindent}{1.9em} % Subfiguras
		\setlength{\cftmyindexequationsindent}{1.9em} % Ecuaciones
		\def\cftlstlistingindent {1.9em} % Códigos fuente
	}{
		\setlength{\cfttabindent}{0em} % Tablas
		\setlength{\cftfigindent}{0em} % Figuras
		\setlength{\cftsubfigindent}{0em} % Subfiguras
		\setlength{\cftmyindexequationsindent}{0em} % Ecuaciones
		\def\cftlstlistingindent {0em} % Códigos fuente
	}
	
	% -------------------------------------------------------------------------
	% Calcula tamaño del margen de los números en objetos del índice
	% -------------------------------------------------------------------------
	% Código fuente
	\ifthenelse{\equal{\showsectioncaptioncode}{none}}{
		\def\cftdefautnumwidthcode {3.0em} % Añade +0.7em
		\def\cftdefaultnumwidthromancode {5.25em} % Añade +0.5em para no overflow
	}{
	\ifthenelse{\equal{\showsectioncaptioncode}{sec}}{
		\def\cftdefautnumwidthcode {3.7em}
		\def\cftdefaultnumwidthromancode {5.75em}
	}{
	\ifthenelse{\equal{\showsectioncaptioncode}{ssec}}{
		\def\cftdefautnumwidthcode {4.4em}
		\def\cftdefaultnumwidthromancode {6.25em}
	}{
	\ifthenelse{\equal{\showsectioncaptioncode}{sssec}}{
		\def\cftdefautnumwidthcode {5.1em}
		\def\cftdefaultnumwidthromancode {6.75em}
	}{
	\ifthenelse{\equal{\showsectioncaptioncode}{ssssec}}{
		\def\cftdefautnumwidthcode {5.8em}
		\def\cftdefaultnumwidthromancode {7.25em}
	}{
	\ifthenelse{\equal{\showsectioncaptioncode}{chap}}{
		\def\cftdefautnumwidthcode {3.0em}
		\def\cftdefaultnumwidthromancode {5.25em}
	}{
		\throwbadconfig{Valor configuracion incorrecto}{\showsectioncaptioncode}{none,chap,sec,ssec,sssec,ssssec}}}}}}
	}
	
	% Código fuente
	\ifthenelse{\equal{\showsectioncaptioneqn}{none}}{
		\def\cftdefautnumwidtheqn {3.0em} % Añade +0.7em
		\def\cftdefaultnumwidthromaneqn {5.25em} % Añade +0.5em para no overflow
	}{
	\ifthenelse{\equal{\showsectioncaptioneqn}{sec}}{
		\def\cftdefautnumwidtheqn {3.7em}
		\def\cftdefaultnumwidthromaneqn {5.75em}
	}{
	\ifthenelse{\equal{\showsectioncaptioneqn}{ssec}}{
		\def\cftdefautnumwidtheqn {4.4em}
		\def\cftdefaultnumwidthromaneqn {6.25em}
	}{
	\ifthenelse{\equal{\showsectioncaptioneqn}{sssec}}{
		\def\cftdefautnumwidtheqn {5.1em}
		\def\cftdefaultnumwidthromaneqn {6.75em}
	}{
	\ifthenelse{\equal{\showsectioncaptioneqn}{ssssec}}{
		\def\cftdefautnumwidtheqn {5.8em}
		\def\cftdefaultnumwidthromaneqn {7.25em}
	}{
	\ifthenelse{\equal{\showsectioncaptioneqn}{chap}}{
		\def\cftdefautnumwidtheqn {3.0em}
		\def\cftdefaultnumwidthromaneqn {5.25em}
	}{
		\throwbadconfig{Valor configuracion incorrecto}{\showsectioncaptioneqn}{none,chap,sec,ssec,sssec,ssssec}}}}}}
	}
	
	% Figuras
	\ifthenelse{\equal{\showsectioncaptionfig}{none}}{
		\def\cftdefautnumwidthfig {3.0em} % Añade +0.7em
		\def\cftdefaultnumwidthromanfig {5.25em} % Añade +0.5em
	}{
	\ifthenelse{\equal{\showsectioncaptionfig}{sec}}{
		\def\cftdefautnumwidthfig {3.7em}
		\def\cftdefaultnumwidthromanfig {5.75em}
	}{
	\ifthenelse{\equal{\showsectioncaptionfig}{ssec}}{
		\def\cftdefautnumwidthfig {4.4em}
		\def\cftdefaultnumwidthromanfig {6.25em}
	}{
	\ifthenelse{\equal{\showsectioncaptionfig}{sssec}}{
		\def\cftdefautnumwidthfig {5.1em}
		\def\cftdefaultnumwidthromanfig {6.75em}
	}{
	\ifthenelse{\equal{\showsectioncaptionfig}{ssssec}}{
		\def\cftdefautnumwidthfig {5.8em}
		\def\cftdefaultnumwidthromanfig {7.25em}
	}{
	\ifthenelse{\equal{\showsectioncaptionfig}{chap}}{
		\def\cftdefautnumwidthfig {3.0em}
		\def\cftdefaultnumwidthromanfig {5.25em}
	}{
		\throwbadconfig{Valor configuracion incorrecto}{\showsectioncaptionfig}{none,chap,sec,ssec,sssec,ssssec}}}}}}
	}
	
	% Tablas
	\ifthenelse{\equal{\showsectioncaptiontab}{none}}{
		\def\cftdefautnumwidthtab {3.0em} % Añade +0.7em
		\def\cftdefaultnumwidthromantab {5.25em} % Añade +0.5em
	}{
	\ifthenelse{\equal{\showsectioncaptiontab}{sec}}{
		\def\cftdefautnumwidthtab {3.7em}
		\def\cftdefaultnumwidthromantab {5.75em}
	}{
	\ifthenelse{\equal{\showsectioncaptiontab}{ssec}}{
		\def\cftdefautnumwidthtab {4.4em}
		\def\cftdefaultnumwidthromantab {6.25em}
	}{
	\ifthenelse{\equal{\showsectioncaptiontab}{sssec}}{
		\def\cftdefautnumwidthtab {5.1em}
		\def\cftdefaultnumwidthromantab {6.75em}
	}{
	\ifthenelse{\equal{\showsectioncaptiontab}{ssssec}}{
		\def\cftdefautnumwidthtab {5.8em}
		\def\cftdefaultnumwidthromantab {7.25em}
	}{
	\ifthenelse{\equal{\showsectioncaptiontab}{chap}}{
		\def\cftdefautnumwidthtab {3.0em}
		\def\cftdefaultnumwidthromantab {5.25em}
	}{
		\throwbadconfig{Valor configuracion incorrecto}{\showsectioncaptiontab}{none,chap,sec,ssec,sssec,ssssec}}}}}}
	}
	
	% Configuración identado de títulos de objetos después del número
	\def\cftfignumwidth {\cftdefautnumwidth}
	% \def\cftsubfignumwidth {\cftdefautnumwidth}
	\def\cfttabnumwidth {\cftdefautnumwidth}
	\def\cftlstlistingnumwidth {\cftdefautnumwidth}
	
	% Código fuente
	\ifthenelse{\equal{\captionnumcode}{arabic}}{ % No hace nada (default)
		\def\cftlstlistingnumwidth {\cftdefautnumwidthcode}
	}{
		\ifthenelse{\equal{\captionnumcode}{roman}}{
			\def\cftlstlistingnumwidth {\cftdefaultnumwidthromancode}
		}{
		\ifthenelse{\equal{\captionnumcode}{Roman}}{
			\def\cftlstlistingnumwidth {\cftdefaultnumwidthromancode}
		}{
			\def\cftlstlistingnumwidth {\cftdefautnumwidthcode}
		}}
	}
	
	% Ecuaciones
	\ifthenelse{\equal{\captionnumequation}{arabic}}{ % No hace nada (default)
		\def\cftmyindexequationsnumwidth {\cftdefautnumwidtheqn}
	}{
		\ifthenelse{\equal{\captionnumequation}{roman}}{
			\def\cftmyindexequationsnumwidth {\cftdefaultnumwidthromaneqn}
		}{
		\ifthenelse{\equal{\captionnumequation}{Roman}}{
			\def\cftmyindexequationsnumwidth {\cftdefaultnumwidthromaneqn}
		}{
			\def\cftmyindexequationsnumwidth {\cftdefautnumwidtheqn}
		}}
	}
	
	% Figuras
	\ifthenelse{\equal{\captionnumfigure}{arabic}}{ % No hace nada (default)
		\def\cftfignumwidth {\cftdefautnumwidthfig}
	}{
		\ifthenelse{\equal{\captionnumfigure}{roman}}{
			\def\cftfignumwidth {\cftdefaultnumwidthromanfig}
		}{
			\ifthenelse{\equal{\captionnumfigure}{Roman}}{
				\def\cftfignumwidth {\cftdefaultnumwidthromanfig}
			}{
				\def\cftfignumwidth {\cftdefautnumwidthfig}
			}}
	}
	
	% Tablas
	\ifthenelse{\equal{\captionnumtable}{arabic}}{ % No hace nada (default)
		\def\cfttabnumwidth {\cftdefautnumwidthtab}
	}{
		\ifthenelse{\equal{\captionnumtable}{roman}}{
			\def\cfttabnumwidth {\cftdefaultnumwidthromantab}
		}{
			\ifthenelse{\equal{\captionnumtable}{Roman}}{
				\def\cfttabnumwidth {\cftdefaultnumwidthromantab}
			}{
				\def\cfttabnumwidth {\cftdefautnumwidthtab}
			}}
	}
	
	% -------------------------------------------------------------------------
	% Genera las funciones para los índices
	% -------------------------------------------------------------------------
	\newcommand{\LoIf}{ % Lista de figuras
		\iftotalfigures
			\ifthenelse{\equal{\indexnewpagef}{true}}{\clearpage}{}
			\ifthenelse{\equal{\addindextobookmarks}{true}}{
				\ifthenelse{\equal{\addindexsubtobookmarks}{true}}{
					\phantomsection
					\belowpdfbookmark{\nameltfigure}{clof}}{}}{
			}
			\listoffigures
		\fi
	}
	\newcommand{\LoIt}{ % Tablas
		\iftotaltables
			\ifthenelse{\equal{\indexnewpaget}{true}}{\clearpage}{}
			\ifthenelse{\equal{\addindextobookmarks}{true}}{
				\ifthenelse{\equal{\addindexsubtobookmarks}{true}}{
					\phantomsection
					\belowpdfbookmark{\namelttable}{clot}}{}}{
			}
			\listoftables
		\fi
	}
	\newcommand{\LoIc}{ % Códigos fuente (listings)
		\iftotallstlistings
			\ifthenelse{\equal{\indexnewpagec}{true}}{\clearpage}{}
			\ifthenelse{\equal{\addindextobookmarks}{true}}{
				\ifthenelse{\equal{\addindexsubtobookmarks}{true}}{
					\phantomsection
					\belowpdfbookmark{\nameltsrc}{clsrc}}{}}{
			}
			\lstlistoflistings
		\fi
	}
	\newcommand{\LoIe}{ % Ecuaciones
		\iftotaltemplateIndexEquationss
			\ifthenelse{\equal{\indexnewpagee}{true}}{\clearpage}{}
			\ifthenelse{\equal{\addindextobookmarks}{true}}{
				\ifthenelse{\equal{\addindexsubtobookmarks}{true}}{
					\phantomsection
					\belowpdfbookmark{\namelteqn}{cleqn}}{}}{
			}
			\listofmyindexequations
		\fi
	}
	
	% -------------------------------------------------------------------------
	% Índice de contenidos
	% -------------------------------------------------------------------------
	\ifthenelse{\equal{\showindexofcontents}{true}}{
		\tableofcontents
	}{}
	
	% -------------------------------------------------------------------------
	% Índice de objetos
	% -------------------------------------------------------------------------
	\ifthenelse{\equal{\indexstyle}{ftc}}{%
		\LoIf\LoIt\LoIc
	}{
	\ifthenelse{\equal{\indexstyle}{}}{%
	}{
	\ifthenelse{\equal{\indexstyle}{e}}{%
		\LoIe
	}{
	\ifthenelse{\equal{\indexstyle}{c}}{%
		\LoIc
	}{
	\ifthenelse{\equal{\indexstyle}{f}}{%
		\LoIf
	}{
	\ifthenelse{\equal{\indexstyle}{t}}{%
		\LoIt
	}{
	\ifthenelse{\equal{\indexstyle}{ec}}{%
		\LoIe\LoIc
	}{
	\ifthenelse{\equal{\indexstyle}{ce}}{%
		\LoIc\LoIe
	}{
	\ifthenelse{\equal{\indexstyle}{ef}}{%
		\LoIe\LoIf
	}{
	\ifthenelse{\equal{\indexstyle}{fe}}{%
		\LoIf\LoIe
	}{
	\ifthenelse{\equal{\indexstyle}{et}}{%
		\LoIe\LoIt
	}{
	\ifthenelse{\equal{\indexstyle}{te}}{%
		\LoIt\LoIe
	}{
	\ifthenelse{\equal{\indexstyle}{cf}}{%
		\LoIc\LoIf
	}{
	\ifthenelse{\equal{\indexstyle}{fc}}{%
		\LoIf\LoIc
	}{
	\ifthenelse{\equal{\indexstyle}{ct}}{%
		\LoIc\LoIt
	}{
	\ifthenelse{\equal{\indexstyle}{tc}}{%
		\LoIt\LoIc
	}{
	\ifthenelse{\equal{\indexstyle}{ft}}{%
		\LoIf\LoIt
	}{
	\ifthenelse{\equal{\indexstyle}{tf}}{%
		\LoIt\LoIf
	}{
	\ifthenelse{\equal{\indexstyle}{ecf}}{%
		\LoIe\LoIc\LoIf
	}{
	\ifthenelse{\equal{\indexstyle}{efc}}{%
		\LoIe\LoIf\LoIc
	}{
	\ifthenelse{\equal{\indexstyle}{cef}}{%
		\LoIc\LoIe\LoIf
	}{
	\ifthenelse{\equal{\indexstyle}{cfe}}{%
		\LoIc\LoIf\LoIe
	}{
	\ifthenelse{\equal{\indexstyle}{fec}}{%
		\LoIf\LoIe\LoIc
	}{
	\ifthenelse{\equal{\indexstyle}{fce}}{%
		\LoIf\LoIc\LoIe
	}{
	\ifthenelse{\equal{\indexstyle}{ect}}{%
		\LoIe\LoIc\LoIt
	}{
	\ifthenelse{\equal{\indexstyle}{etc}}{%
		\LoIe\LoIt\LoIc
	}{
	\ifthenelse{\equal{\indexstyle}{cet}}{%
		\LoIc\LoIe\LoIt
	}{
	\ifthenelse{\equal{\indexstyle}{cte}}{%
		\LoIc\LoIt\LoIe
	}{
	\ifthenelse{\equal{\indexstyle}{tec}}{%
		\LoIt\LoIe\LoIc
	}{
	\ifthenelse{\equal{\indexstyle}{tce}}{%
		\LoIt\LoIc\LoIe
	}{
	\ifthenelse{\equal{\indexstyle}{eft}}{%
		\LoIe\LoIf\LoIt
	}{
	\ifthenelse{\equal{\indexstyle}{etf}}{%
		\LoIe\LoIt\LoIf
	}{
	\ifthenelse{\equal{\indexstyle}{fet}}{%
		\LoIf\LoIe\LoIt
	}{
	\ifthenelse{\equal{\indexstyle}{fte}}{%
		\LoIf\LoIt\LoIe
	}{
	\ifthenelse{\equal{\indexstyle}{tef}}{%
		\LoIt\LoIe\LoIf
	}{
	\ifthenelse{\equal{\indexstyle}{tfe}}{%
		\LoIt\LoIf\LoIe
	}{
	\ifthenelse{\equal{\indexstyle}{cft}}{%
		\LoIc\LoIf\LoIt
	}{
	\ifthenelse{\equal{\indexstyle}{ctf}}{%
		\LoIc\LoIt\LoIf
	}{
	\ifthenelse{\equal{\indexstyle}{fct}}{%
		\LoIf\LoIc\LoIt
	}{
	\ifthenelse{\equal{\indexstyle}{tcf}}{%
		\LoIt\LoIc\LoIf
	}{
	\ifthenelse{\equal{\indexstyle}{tfc}}{%
		\LoIt\LoIf\LoIc
	}{
	\ifthenelse{\equal{\indexstyle}{ecft}}{%
		\LoIe\LoIc\LoIf\LoIt
	}{
	\ifthenelse{\equal{\indexstyle}{ectf}}{%
		\LoIe\LoIc\LoIt\LoIf
	}{
	\ifthenelse{\equal{\indexstyle}{efct}}{%
		\LoIe\LoIf\LoIc\LoIt
	}{
	\ifthenelse{\equal{\indexstyle}{eftc}}{%
		\LoIe\LoIf\LoIt\LoIc
	}{
	\ifthenelse{\equal{\indexstyle}{etcf}}{%
		\LoIe\LoIt\LoIc\LoIf
	}{
	\ifthenelse{\equal{\indexstyle}{etfc}}{%
		\LoIe\LoIt\LoIf\LoIc
	}{
	\ifthenelse{\equal{\indexstyle}{ceft}}{%
		\LoIc\LoIe\LoIf\LoIt
	}{
	\ifthenelse{\equal{\indexstyle}{cetf}}{%
		\LoIc\LoIe\LoIt\LoIf
	}{
	\ifthenelse{\equal{\indexstyle}{cfet}}{%
		\LoIc\LoIf\LoIe\LoIt
	}{
	\ifthenelse{\equal{\indexstyle}{cfte}}{%
		\LoIc\LoIf\LoIt\LoIe
	}{
	\ifthenelse{\equal{\indexstyle}{ctef}}{%
		\LoIc\LoIt\LoIe\LoIf
	}{
	\ifthenelse{\equal{\indexstyle}{ctfe}}{%
		\LoIc\LoIt\LoIf\LoIe
	}{
	\ifthenelse{\equal{\indexstyle}{fect}}{%
		\LoIf\LoIe\LoIc\LoIt
	}{
	\ifthenelse{\equal{\indexstyle}{fetc}}{%
		\LoIf\LoIe\LoIt\LoIc
	}{
	\ifthenelse{\equal{\indexstyle}{fcet}}{%
		\LoIf\LoIc\LoIe\LoIt
	}{
	\ifthenelse{\equal{\indexstyle}{fcte}}{%
		\LoIf\LoIc\LoIt\LoIe
	}{
	\ifthenelse{\equal{\indexstyle}{ftec}}{%
		\LoIf\LoIt\LoIe\LoIc
	}{
	\ifthenelse{\equal{\indexstyle}{ftce}}{%
		\LoIf\LoIt\LoIc\LoIe
	}{
	\ifthenelse{\equal{\indexstyle}{tecf}}{%
		\LoIt\LoIe\LoIc\LoIf
	}{
	\ifthenelse{\equal{\indexstyle}{tefc}}{%
		\LoIt\LoIe\LoIf\LoIc
	}{
	\ifthenelse{\equal{\indexstyle}{tcef}}{%
		\LoIt\LoIc\LoIe\LoIf
	}{
	\ifthenelse{\equal{\indexstyle}{tcfe}}{%
		\LoIt\LoIc\LoIf\LoIe
	}{
	\ifthenelse{\equal{\indexstyle}{tfec}}{%
		\LoIt\LoIf\LoIe\LoIc
	}{
	\ifthenelse{\equal{\indexstyle}{tfce}}{%
		\LoIt\LoIf\LoIc\LoIe
	}{
		\throwbadconfig{Estilo desconocido del indice}{\indexstyle}{ftc,,e,c,f,t,ec,ce,ef,fe,et,te,cf,fc,ct,tc,ft,tf,ecf,efc,cef,cfe,fec,fce,ect,etc,cet,cte,tec,tce,eft,etf,fet,fte,tef,tfe,cft,ctf,fct,tcf,tfc,ecft,ectf,efct,eftc,etcf,etfc,ceft,cetf,cfet,cfte,ctef,ctfe,fect,fetc,fcet,fcte,ftec,ftce,tecf,tefc,tcef,tcfe,tfec,tfce}}}}}}}}}}}}}}}}}}}}}}}}}}}}}}}}}}}}}}}}}}}}}}}}}}}}}}}}}}}}}}}}}
	}
	
	% -------------------------------------------------------------------------
	% Termina el bloque de índice
	% -------------------------------------------------------------------------
	\sectionfont{\color{\sectioncolor} \sectionfontsize \sectionfontstyle \selectfont}
	\endgroup
	
}

% END